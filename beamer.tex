\documentclass{beamer}
%
% Choose how your presentation looks.
%
% For more themes, color themes and font themes, see:
% http://deic.uab.es/~iblanes/beamer_gallery/index_by_theme.html
%
\mode<presentation>
{
  \usetheme{default}      % or try Darmstadt, Madrid, Warsaw, ...
  \usecolortheme{default} % or try albatross, beaver, crane, ...
  \usefonttheme{default}  % or try serif, structurebold, ...
  \setbeamertemplate{navigation symbols}{}
  \setbeamertemplate{caption}[numbered]
} 

\usepackage[english]{babel}
\usepackage[utf8x]{inputenc}

\title[Your Short Title]{Probabilidade e Estatística}
\author{José William Vitorino de Souza}
\institute{IFCE - Aracati}
\date{19/02/2018}

\begin{document}

\begin{frame}
  \titlepage
\end{frame}

 %Uncomment these lines for an automatically generated outline.
%\begin{frame}{Outline}
  %\tableofcontents
%\end{frame}

\section{Introdução}

\begin{frame}{Introdução}

%\vskip 1cm

\begin{block}{}
O trabalho consiste em responder duas questões. Uma de cada matéria a seguir:
\end{block}

\begin{itemize}
  \item Probabilidade.
  \item Contagem.
\end{itemize}

\end{frame}

\section{Some \LaTeX{} Examples}

\subsection{Questão 1}

\begin{frame}{Questão 1}

\textbf{(O Problema do Bode)} Este problema foi proposto em um programa de radio nos Estados Unidos e causou um enorme debate na internet. Em um programa de premios, o candidato tem diante de si tres portas. Atrâs de uma dessas portas, ha um grande prêmio; atrâs das demais há um bode. O candidato escolhe inicialmente uma das portas. O apresentador (que sabe qual e a porta que contém o prêmio) abre uma das portas nao indicadas pelo candidato, mostrando necessariamente um bode. A seguir, ele pergunta se o candidato mantem sua escolha ou deseja trocar de porta. O candidato deve trocar ou nao?

\end{frame}

\begin{frame}{Resposta}

São 3 portas, vamos chamar de portas 1, 2 e 3. Vamos admitir que a porta que tem o prêmio seja a porta 3.

Sem saber qual é a certa, vamos simular as possíveis escolhas com as possíveis decisões.

\end{frame}

\begin{frame}
	Caso 1 - Trocando de porta:
	\begin{enumerate}
		\item Escolhe porta 1, abre porta 2, escolhe porta 3 e ganha!
		\item Escolhe porta 2, abre porta 1, escolhe porta 3 e ganha!
		\item Escolhe porta 3, abre porta 1, escolhe porta 2 e não ganha!
	\end{enumerate}
	 Probabilidade de ganhar: $${2\over 3}$$
\end{frame}

\begin{frame}
	Caso 2 - Não trocando de porta:
	\begin{enumerate}
		\item Escolhe porta 1, abre porta 2, continua na porta 1 e não ganha!
		\item Escolhe porta 2, abre porta 1, continua na porta 2 e não ganha!
		\item Escolhe porta 3, abre porta 1, continua na porta 3 e ganha!
	\end{enumerate}
	 Probabilidade de ganhar: $${1\over 3}$$
\end{frame}

\subsection{Questão 2}

\begin{frame}{Questão 2}

%Let $X_1, X_2, \ldots, X_n$ be a sequence of independent and identically distributed random variables with $\text{E}[X_i] = \mu$ and $\text{Var}[X_i] = \sigma^2 < \infty$, and let
%$$S_n = \frac{X_1 + X_2 + \cdots + X_n}{n}
 %     = \frac{1}{n}\sum_{i}^{n} X_i$$
%denote their mean. Then as $n$ approaches infinity, the random variables $\sqrt{n}(S_n - \mu)$ converge in distribution to a normal $\mathcal{N}(0, \sigma^2)$.

\end{frame}

\end{document}
